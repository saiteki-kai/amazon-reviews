\section{Dataset}
Il dataset utilizzato in questo lavoro contiene i prodotti e le recensioni della sottocategoria \textit{Internal Computer Components} di Amazon dal 1999 fino al 2018.

Nella tabella X sono descritte le caratteristiche del dataset. % TODO: sistemare (attrs, avg reviews per product)

% TODO: aggiungere
Per prima cosa è stata effettuata una pulizia dei dati per alcuni attributi.
%
Brand Cleaning

Sono stati identificati i valori mancati e i valori duplicati di prodotti e recensioni. 
Infine sono stati tenute solo i prodotti che hanno almeno una recensione associata.

Nella figura \ref{fig:missing_values} sono riportate le percentuali di valori mancanti. Le uniche recensioni che sono state 
scartate sono quelle con testo vuoto che sono solo 2.

% TODO: aggiungere numero duplicati

\begin{figure}[ht]
  \centering
  \includegraphics[width=0.95\textwidth]{images/dataset/missing_values.png}
  \caption{Percentuale dei valori mancanti per ogni attributo.}
  \label{fig:missing_values}
\end{figure}

Tramite il modello FastText \cite{joulin2016bag} sono state identificate e tenute 
solo le recensioni in lingua inglese con probabilità superiore o uguale al 95\%.

% TODO: riscrivere meglio
In input sono stati passati al modello i campi \textit{summary} e \textit{text} concatenati tra loro e su questo nuovo 
campo sono stati rimossi gli URL e i tag HTML e infine trasformato in minuscolo.

Il numero di recensioni rimanenti è 183,856 che corrisponde al 68\% del dataset.

% TODO: controllare cbe siano giusti data l'analisi divisa tra prodotti e reviews...

\subsection{Analisi}
Successivamente è stata effettuata un'analisi esplorativa del dataset pulito.
