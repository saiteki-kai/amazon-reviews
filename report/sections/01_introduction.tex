\section{Introduzione}
In questo progetto è stato analizzato uno dei dataset disponibili riguardante le recensioni di prodotti Amazon \cite{ni2019justifying}.

Al progetto è stato dato un taglio ben preciso, ovvero ci siamo messi nella condizione in cui un brand avesse contattato la nostra azienda per eseguire delle indagini in grado di rappresentare nel modo più fedele possibile l'esperienza del pubblico.

Per soddisfare la richiesta abbiamo cercato di rispondere alle seguenti domande in modo da delineare una linea guida per le analisi: 

\begin{itemize}
\item Qual'è il gradimento del pubblico in merito al brand ?
\item Sono presenti prodotti non apprezzati dal pubblico ?
\item Quali sono gli aspetti che caratterizzano il brand e come sono percepiti dal pubblico ?  
\item Quali sono i principali competitors ? 
\item Come il brand è percepito rispetto ai competitors ?
\item Un confronto degli aspetti che caratterizzano il brand rispetto ai competitors.
\label{domande-iniziali}
\end{itemize}

Il brand scelto è \textbf{Corsair} leader nella produzione di componenti per computer come schede RAM, memorie, alimentatori e sistemi di raffreddamento.

Nei successivi capitoli verranno descritte le scelte progettuali che hanno permesso l'analisi aspect-based sentiment dei dati in merito alla linea guida scelta.

Inoltre verranno mostrate le tecniche utilizzate per gestire le problematiche introdotte dal linguaggio naturale e le decisioni prese per poter caratterizzare in modo corretto le recensioni evitando di introdurre artefatti non riconducibili al brand, ma al servizio offerto da Amazon, i quali possono distorcere le conclusioni tratte dalle analisi.
