\section{Conclusioni}
Dopo aver eseguito un pre-processing sul testo, un'analisi iniziale sui dati e diversi esperimenti è stato possibile eseguire delle analisi aspect-based dei dati. Questo ci ha permesso di trovare risposta alle domande poste per poter compiere al meglio la richiesta posta dal brand nostro cliente, ovvero Corsair.

In sintesi quello che possiamo dire è che negli utenti si riscontra un alto gradimento verso il brand e i suoi prodotti, nei quali non è stato riscontrato nessun prodotto che danneggi l'immagine della Corsair.

I principali aspetti associati al brand e il relativo gradimento da parte del pubblico permettono di identificare i punti di forza e cosa più importante permette di identificare gli aspetti critici che richiedono più attenzione da parte di Corsair.

Grazie alle analisi è stato possibile identificare quali sono i principali competitor e analizzarne il \textit{sentiment} rispetto al nostro brand. Sia rispetto ad una valutazione generale, sia rispetto a una valutazione dettaglia sul singolo topic.

In questo progetto due modelli ASUM e JST per l'aspect-based sentiment analysis sono stati addestrati e comparati con diverse prove effettuando il task di sentiment analysis. I risultati migliori sono stati ottenuti da ASUM considerando 50 topic ed effettuando lo \textit{stemming}, con un valore di F1 del 63.8\%.

Un'ulteriore prova è stata effettuata per cercare di migliorare la fase di negazione dei token, suddividendo le frasi dove presenti delle congiunzioni avversative, ma che non ha portato alcun miglioramento.

Sono molti i possibili miglioramenti tra cui diversi pre-processing, in particolare migliorare la suddivisione delle frasi permetterebbe una migliore negazione dei token aumentando le performance del riconoscimento del \textit{sentiment}, inoltre si potrebbe considerare il \textit{sentiment} neutrale, espandere i \textit{seed} di partenza ed effettuare una valutazione dei topic estratti con una \textit{groundtruth}.
